\documentclass[paper=a4, fontsize=11pt]{article}

%code syntax highliter
\usepackage[usenames,dvipsnames]{color,xcolor}
\usepackage{listings}

\usepackage{setspace}
\usepackage{wrapfig}
\usepackage{caption}
\usepackage{mathtools}
\usepackage[hidelinks]{hyperref}
\usepackage{fancyhdr}
\usepackage{array} % "m'' used in tabular env.

\usepackage{tikz}
\usetikzlibrary{shapes.geometric}
\usetikzlibrary{calc}

\usepackage{pgfplots}
\pgfplotsset{compat=1.13} % Nicer axis label placement

\usepackage{cleveref}
\usepackage{graphicx}
\graphicspath{ {../../../formats/globals/} {images/}}

\usepackage{tikz,lmodern}
\usepackage[most]{tcolorbox}

\usepackage{xepersian}
\settextfont{Vazir}
\setlatintextfont{Vazir}

\newcommand{\diam}[1]{\tikz\draw (0,0)node[shape aspect=1,diamond,draw,,inner sep=2pt]{#1};} % Diamond



%%%% START‌ Box %%%%
\newenvironment{questionbox}{
\begin{tcolorbox}[colback=red!5!white,colframe=red!75!black]
}{
\end{tcolorbox}
}
%%%% END‌ Box %%%%

%%%%% Start Pages Config %%%%%%
\pagestyle{fancy}
\rhead{ پاسخ تمارین | مبانی هوش }
\lhead{ علی فرجی}
\cfoot{ \diam{\thepage} }
%%%%% End Pages Config %%%%%%

%%%%%% syntax highliter settings %%%%
\lstdefinestyle{customc}{
  belowcaptionskip=1\baselineskip,
  breaklines=true,
  frame=L,
  xleftmargin=\parindent,
  language=C,
numbers=left,
showspaces=false,           % Leerzeichen anzeigen ?
showtabs=false,             % Tabs anzeigen ?
  showstringspaces=false,
xleftmargin=25pt,
xrightmargin=10pt,
framexleftmargin=17pt,
framexrightmargin=5pt,
framexbottommargin=4pt,
  basicstyle=\footnotesize\ttfamily\setstretch{.8},
  keywordstyle=\bfseries\color{green!40!black},
  commentstyle=\itshape\color{purple!40!black},
  identifierstyle=\color{blue},
  stringstyle=\color{orange},
}
%%%%%% syntax highliter settings %%%%

\begin{document}
\thispagestyle{empty}
\setstretch{2}
	\begin{titlepage}
		\begin{center}
		
       \vspace*{1cm}
       
       \textbf{مبانی و کاربرد های هوش مصنوعی}

       \vspace{0.5cm}
       
پاسخ تمارین سری اول
       \vspace{2cm}
       

	\textbf{علی فرجی 9634024}
       
       \vspace{2cm}
       
استاد درس: دکتر سیده فاطمه موسوی
 \vfill

     \includegraphics[width=0.4\textwidth]{aut}
 
دانشگاه صنعتی امیرکبیر (پلی تکنیک تهران) \\
دانشکده مهندسی کامپیوتر\\
\today

   \end{center}
\end{titlepage}

\tableofcontents
\thispagestyle{empty}

\newpage

\setcounter{page}{1}

\section{سوال اول}
\subsection{بخش الف}
بله، میتواند رفتار منطقی داشته باشد به این صورت که به چهار طرف نگاه کند و از بین انها که چاله نیست یکی را به طور اتفاقی انتخاب کنید در همه حالات یک راه حلی وجود خواهد داشت و هیچ موقع در چاله نمی افتد چون در بدترین صورت خانه ای که از آن امده است چاله نبوده و به خانه قبلی بر می گردد.

البته این عامل باید یک اقدام هیچ کاری نکن هم داشته باشد که معیار کارایی نیافتادن در چاله را بیشینه می کنید چون ممکن است از جایی شروع کند که همه طرف آن چاله است و هیچ عملی نکردن در اینجا باعث می شود در چاله نیافتد.
\subsection{بخش ب}
اگر در این حالت فرض کنیم که به چاله نیافتادن به علاوه گنج پیدا کردن بشود هدف ما پس این عامل لزوما عقلانی رفتار نمی کند.

چون به طور مثال ممکن است سمت چپ چاله و سه طرف دیگر، سطح بدون چاله باشد که گنجی ندارند و خانه بالایی به گنج نزدیکتر باشد، این عامل لزوما خانه بالایی را انتخاب نخواهد کرد و ممکن است به سمت راست برود که از گنج دورتر هم می شود.
\subsection{بخش د}
نکته جالب اینجاست که بخش ج نداریم!!

چون عامل تمام صفحه را نمی تواند ببیند پس می توان عاملی مبتنی بر سودمندی طراحی کرد که هر خانه ای را که می رود به حافظه بسپارد و اگر به جایی رسید که  از بین 3 خانه مثلا دوتایش را قبلا رفته است، اولویت رفتن را به خانه ای بدهد که قبلا به سمت آن نرفته است این کار باعث می شود که عامل به مرور زمان تمام صفحه را بگردد و گنج ها را پیدا کند.

چرا این عامل مبتنی بر سودمندی می شود؟

چون مکان هایی که قبلنارفته است را دیده و اگر گنجی بوده، آن را پیدا کرده، پس انتخاب تصادفی از بین همان مسیرها با احتمال کمتری ما را به گنج می رساند و مسیر جدید برای ما سودمند تر است. در اصل ما اینجا بین دو هدف نیافتادن در چاله و بهتر کردن انتخاب سطوح غیر چاله یک قانون سودمندی تعریف کردیم که برای هر سطح اطراف، یک عدد اولویت نسبت می دهد.

یک نکته که باید اینجا عرض کنم این است که در اینجا ما به طور ضمنی زمان و تعداد اکشن ها را برای معیار کارایی در نظر گفته ایم و گرنه همان انتخاب تصادفی از بین مسیرها در زمان بینهایت، ما را به تمام گنج ها می رساند. (اگر راهی وجود داشته باشد)

\section{سوال دوم}
\subsection{بخش الف}
اگر فرض کنیم که عامل ها از انتخاب های همدیگر خبر ندارند و معیار کارایی امتیاز بیشتر گرفتن نسبت به دیگری باشد پس همیشه ما باید B را انتخاب کنیم.

چون انتخاب A دو نتیجه با احتمال یکسان را پیش رو دارد: یا امتیازها مساوی می شود یا 3 امتیاز عقب می افتیم. \\
حال انتخاب B  هم دو نتیجه با احتمال یکسان را پیش رو دارد: یا امتیازها مساوی می شود یا 3 امتیاز جلو می افتیم.

پس بهتر است همیشه حالت مساوی یا بیشتر را برداریم تا به معیار کارایی نزدیک شویم.
امتیاز قابل حصول در هر مرحله 1 یا 5 است که خب اگر عامل دیگر به طور مساوی بین A و B انتخاب کنید در بینهایت به طور میانگین در هر مرحله ما «سه» امتیاز کسب می کنیم و به طور میانگین «یک و نیم» امتیاز در هر مرحله جلو می افتیم.

\subsection{بخش ب}
در حالت کلی فرقی نمی کند که ما بدانیم یا ندانیم که طرف مقابل کدام را انتخاب می کند، زیرا در هر دو صورت ما با انتخاب B می توانیم معیار کارایی را ارضا کنیم و این سوال هم مانند بخش الف خواهد شد.


\section{سوال سوم}


\small
\begin{center}
 \begin{tabular}{|| m{0.2\textwidth} | m{0.15\textwidth} | m{0.15\textwidth} | m{0.15\textwidth} | m{0.15\textwidth} ||} 
 \hline
  --   & معیار کارایی & محیط & عملگرها & حسگرها \\ [0.5ex] 
 \hline\hline
دستیار صوتی اپل & درست و به موقع انجام دادن دستور گفته شده & محیط گوشی برای کار با آن مثلا مخاطبین و ... حتی درایور میکروفون که سنسور به آن وصل می شود & دستورات ارسالی به سیستم & مثلا تابعی که سیگنال صوتی را از میکروفون میگیرد \\
 \hline
پیشنهاد دهنده دوست & موفقیت و رضایت در رابطه & (پایگاه داده) اطلاعات افراد & جستوجو در پایگاه داده & فرم ثبت مشخصات متقاضی \\
 \hline
\end{tabular}
\end{center}
\normalsize

\small
\begin{center}
 \begin{tabular}{|| m{0.15\textwidth} | m{0.1\textwidth} |m{0.1\textwidth} |m{0.1\textwidth} |m{0.1\textwidth} |m{0.1\textwidth} |m{0.1\textwidth} |m{0.1\textwidth} ||} 
 \hline
  --   & قابل مشاهده & تک عاملی/ چندعاملی & قطعی / غیر قطعی & مرحله ای / ترتیبی & ایستا / پویا & گسسته / پیوسته & شناخته / ناشناخته \\ [0.5ex] 
 \hline\hline
دستیار صوتی اپل & نیمه مشاهده پذیر & تک عاملی & غیر قطعی & مرحله ای & پویا & پیوسته & ناشناخته \\
 \hline
پیشنهاد دهنده دوست & قابل مشاهده & تک عاملی & غیر قطعی & ترتیبی & ایستا & گسسته & شناخته \\
 \hline
\end{tabular}
\end{center}
\normalsize

\subsection{توضیحات محیط  دستیار صوتی اپل}
\begin{itemize}
\item
نیمه مشاهده پذیر: چون صدا می تواند نویز داشته باشد پس نیمه مشاهده پذیر است.
\item
تک عاملی: هیچ موجود دیگری معیار کارایی ندارد که بر روی معیار کارایی ما تاثیری بگذارد.
\item
غیر قطعی: مثلا اگر دستور تماس داده شود و دستیار فرایند تماس را اجرا کند معلوم نیست که آنتن دهی درستی در منطقه باشد و اصلا تماس بر قرار شود یا نه و یا اصلا شارژ تمام شود یا برنامه ای کرش کند و بیرون بیافتد.
\item
مرحله ای: در هر مرحله دستور را میگیرد و اجرا می کند مثلا دکمه بالا را بزن. حال این تصمیم ربطی به قبلی که چه چیزی را باز کرده ندارد و می تواند مستقلا این دستور را پردازش و اجرا کند.
\item
پویا: مثلا اگر دستور تماس با فردی آمده و در فاصله ای که میخواهد تماس بگیرد آن فرد خودش زنگ می زند و محیط تغییر میکند. پس پویاست.
\item
پیوسته: سیگنال صوتی به صورت پیوسته است.( مانند سیگنال تصویر که در مثال ها پیوسته در نظر می گرفتیم)
\item
ناشناخته: محیط ناشناخته است زیرا مثلا دستور اجرای فلان برنامه می اید و دستیار باید بررسی کند و یاد بگیرد که اصلا آن برنامه نصب هست یا نه.

\end{itemize}

\subsection{توضیحات محیط  پیشنهاد دهنده دوست}
\begin{itemize}
\item
قابل مشاهده: ورودی شامل فرم دریافتی و پایگاه داده و همه چی برای نرم افزار قابل مشاهده است.
\item
تک عاملی: هیچ موجود دیگری بر روی معیار کارایی ما تاثیر گذار نیست.
\item
غیر قطعی: وضعیت بعدی بر اساس پیشنهاد و ورودی مشخص نیست و ممکن است کاربر بهترین مورد پیشنهادی را قبول نکند.
\item
ترتیبی: اطلاعات وارد شده در هر مرحله از نرم افزار در مراحل بعدی تاثیرگذار است و ممکن است سوال هایی که پرسیده می شود بر طبق مراحل قبلی فرق کند.
\item
ایستا: اگر دیتابیس افراد ثابت باشد یعنی در حین تصمیم گیری دیتابیس اپدیت نشود پس ایستاست.
\item
ورودی ها کلمات و گزینها و در حالت کلی گسسته اند.
\item
شناخته: نتایج برای هر عمل مشخص است که بهترین کیس منطبق را به کاربر پیشنهاد می دهد.
\end{itemize}

\section{سوال چهارم}
عامل ما مبتنی بر یادگیری است.

\begin{itemize}
	\item
محیط: یک صفحه که با یک دوربین به آن نگاه می کند و یک نمایشگر که در کنار دوربین که نتیجه را در روی آن نمایش می دهد.
	\item
حسگرها: یک دوربین.
	\item
عنصر اجرایی: در آوردن ویژگی های صورت مانند چشم و گوش و بینی و مطابقت تشابه هر قسمت با پایگاه داده چهره ها.
	\item
معیار کارایی: درصد تشخیص درست در طول زمان.
	\item
فیدبک: درستی یا نادرستی تصویر شناسایی شده و اگر در پایگاه داده قبل ثبت شده است مشخص کردن این که این چه کسی بود.
	\item
عنصر یادگیری: اگر فیدبک داده شده این گونه بود که یک فرد که در پایگاه داده موجود بود را تشخیص ندادیم باید به گونه ای عمل کنیم که در پایگاه داده ضرایب شخص را تصحیح کنیم تا بر اساس الگوریتم تشخیص به تصویری که اکنون گفتیم نادرست است نزدیک شود و از سری بعد درست تشخیص دهد.
	\item
اهداف یادگیری: نزدیک کردن ضریب تشابه اجزای صورت هر فرد به واقعیتش در حالات مختلف که به درستی تشخیص دهد.
	\item
دانش: ویژگی های صورت هر فرد و اطلاعات هویتی
	\item
تغییرات: تغییرات ضرایب  و ویژگی های چهره هر فرد که در حالات مختلف دارد.
	\item
عملگرها: نمایش پیغام شناسایی شد/ نشد و در صورت شناسایی نام و نام خانوادگی و اطلاعات فرد نمایش داده شود.

\end{itemize}


\end{document}




















































